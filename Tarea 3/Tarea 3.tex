\documentclass[conference]{IEEEtran}
\IEEEoverridecommandlockouts
% The preceding line is only needed to identify funding in the first footnote. If that is unneeded, please comment it out.
\usepackage{cite}
\usepackage{amsmath,amssymb,amsfonts}
\usepackage{algorithmic}
\usepackage{graphicx}
\usepackage{textcomp}
\usepackage{xcolor,multirow}
\def\BibTeX{{\rm B\kern-.05em{\sc i\kern-.025em b}\kern-.08em
    T\kern-.1667em\lower.7ex\hbox{E}\kern-.125emX}}
    
\renewcommand{\abstractname}{Resumen}
\begin{document}

\title{Cálculo de la elasticidad económica a partir de la interpolación de Newton}

\author{
$\;$\\
\IEEEauthorblockN{Stephanie Álvarez Carazo, Juan Peña Rostrán, Ignacio Carazo Nieto, Gabriel González Houdelath}
\IEEEauthorblockA{\textit{Ingeniería en Computadores} \\
\textit{Instituto Tecnológico de Costa Rica}\\
Cartago, 30101, Costa Rica \\
\{correo\_est1, correo\_est2, correo\_est3, correo\_est4, gabogh99@estudiantec.cr\}}
}

\maketitle

\begin{abstract}
Resumen
\end{abstract}


\section{Introducción}
Los métodos numéricos son una parte fundamental de las matemáticas debido a su amplia cantidad de aplicaciones. Gracias a estos, es posible generar modelos matemáticos computacionales para resolver problemas específicos de todo tipo de complejidad y aplicados a prácticamente todas las área de la vida. Uno de los métodos más conocidos y utilizados es el método de interpolación polinomial, la cual permite calcular valores intermedios entre datos definidos por puntos, teniendo un solo polinomio de grado n que se ajusta a los n+1 puntos conocidos. (Chapra,S; Canale, R, 2006). De manera más específica, el método a utilizar será la interpolación polinomial de Newton, la cual aplica las diferencias divididas para encontrar un polinomio de orden n corrrespondiente a los n+1 puntos conocidos.

Para la realización de este trabajo se aplicará dicho método numérico con la econometría, área de la economía que se encarga de la elaboración de modelos matemáticos para explicación, análisis y predicción de los diferentes hechos económicos que se producen en los diferentes mercados mundiales. A partir de dichos mercados, para cuyo análisis es necesario conocer una cantidad limitada de datos, es que se puede generar la relación con la interpolación polinomial; representando los puntos conocidos en el método de interpolación como las variables exógenas y endógenas conocidas y, en la mayoría de casos, limitadas que alteran los mercados como ventas, producción, precios, entre otras, por lo que la interpolación polinomial de Newton resulta un método efectivo para solucionar dicho problema.

La relevancia del presente trabajo radica en la importancia del análisis de los mercados a todo nivel social, tanto para una empresa como una nación es fundamental un correcto estudio de los mercados en los cuales se desarrollan para su éxito económico. 


Adentrándose en el método de Newton para el cálculo de polinomios de grado n que permitan conocer el estudio de los mercados como fue mencionado anteriormente ....



\section{Objetivos}

\subsection{Objetivo General}
Objetivo General

\subsection{Objetivos Específicos}
Objetivos Específicos

\section{Metodología}
Metodología

\section{Resultados Numéricos}
Resutados Numéricos

\section{Bibliografía}
Bibliografía


\end{document}